\section{Programos testavimo protokolas}

    Testavimui pasirinkta juodos dežės metodas. 
    Testavimui pasirinkintas įvairus spektras mūsų pačių sugalvotų ribų. 
	Stengtasi padengti visas palaikomas Teiloro skleidinių funkcijas.
	Taip pat, mėginome ištestuoti skirtingas funkcijų kompozicijas.
	
	Žemiau pateiktoje \ref{tab:rez} lentelėje vaizduojami visi testavimo atvejai, sugrupuoti į dvi grupes - 
	elementariųjų funkcijų bei jų kompozicijų ribos. 
	Be to, pateikiamas tikėtini ir gauti ribų atsakimų rezultatai. 
	Jeigu programos gautasis atsakymas sutapo su tikrąja funkcijos riba, 
    rezultatas teigiamas ir pažymėtas žalia spalva, kitu atveju neigiamas - raudona.
    
\begin{center}
    \newcommand*\rot{\rotatebox{90}}
    \newcommand{\x}{$\times$}
    \newcommand{\green}{\cellcolor{green!50} teigiamas}
    \newcommand{\red}{\cellcolor{red!50} neigiamas}
    \newcommand{\noLimit}{Riba neegzistuoja}
    \newcommand{~}{$\approx$}
    \newcommand{\header}[2]{\textbf{#1} & \multicolumn{3}{c}{\textbf{#2}} & \\ \hline}
    \begin{longtable}{|m{0.09\textwidth}|m{0.4\textwidth}|m{0.12\textwidth}|m{0.11\textwidth}|m{0.11\textwidth}|}
    \caption{Testavimo rezultatai} \label{tab:rez} \\
    \hline
        \textbf{Testo numeris} & \textbf{Skaičiuojama riba} &\textbf{Tikėtinas rezultatas} & \textbf{Gautas rezultatas} & \textbf{Rezultatas} \\ \hline
    \endhead
        \multicolumn{5}{|r|}{{Tęsinys kitame puslapyje}} \\ \hline
    \endfoot
    \hline \hline
    \endlastfoot
    \header{0}{pvz}
    0.1 & lim & 1 & 2  & \green \\ \hline
    0.2 & lim & 1 & 2  & \red \\ \hline

    \header{1}{Elementarių funkcijų ribos}
    1.1 &\[ \lim_{x \to -1} \frac{1}{1+x} \]                                                    & \noLimit  &           &           \\ \hline
    1.2 & \[ \lim_{x \to 2} \frac{2^x}{x^2} \]                                                  & 1         &           &           \\ \hline
    1.3 & \[\lim_{x \to 30} \sqrt[5]{x}\]                                                       & ~1.97     & ~1.97     & \green    \\ \hline
    1.4 & \[ \lim_{x \to 1.57} cos(2x) \]                                                       & ~-1       &           &           \\ \hline
    1.5 & \[ \lim_{x \to 0} e^{\frac{1}{x}} \]                                                  & $+\infty$ &           &           \\ \hline
    1.6 & \[ \lim_{x \to 0} \frac{sin(x)}{x} \]                                                 & 1         &           &           \\ \hline
    1.7 & \[ \lim_{x \to \infty} \frac{sin(x)}{x}\]                                             & 0         &           &           \\ \hline
    1.8 & \[ \lim_{x \to 0} \frac{sin(x)}{cos(x)} \]                                            & 0         &           &           \\ \hline
    1.9 & \[ \lim_{x \to 0} \frac{cos(x)}{sin(x)} \]                                            & \noLimit  &           &           \\ \hline    
    1.10 & \[ \lim_{x \to \infty} \frac{1}{x} \]                                                & 0         &           &           \\ \hline
    1.11 & \[ \lim_{x \to -\infty} \frac{x^2}{x^4} \]                                           & 0         &           &           \\ \hline
    1.12 & \[ \lim_{x \to \infty} \frac{x^{\frac{3}{2}}}{x(\sqrt{x+1}+\sqrt{x-1}+2\sqrt{x})} \] & 1/4       &           &           \\ \hline
    1.13 & \[ \lim_{x \to 0} \frac{e^xsin(x)-x(1+x)}{x^3} \]                                    & 1/3       & 1/3       & \green    \\ \hline
    1.14 & \[ \lim_{x \to 0} \frac{cos(x)-e^{(-\frac{x^2}{2})}}{x^4} \]                         & -1/12     &           &           \\ \hline
    1.15 & \[ \lim_{x \to 5} (5+x)^5 \]                                                         & 100000    &           &           \\ \hline
    1.16 & \[ \lim_{x \to 0.7854} \sqrt(2)sin(x) \]                                             & ~1        &           &           \\ \hline
    1.17 & \[ \lim_{x \to 1} \frac{x^{\frac{3}{2}}}{x(\sqrt{x+1}+\sqrt{x-1}+2\sqrt{x})} \]      & ~0.2929   &           &           \\ \hline
    1.18 & \[ \lim_{x \to \infty} \frac{x^2}{e^x} \]                                            & 0         &           &           \\ \hline
    1.19 & \[ \lim_{x \to 0} \frac{\sqrt{x+4}-2}{x} \]                                          & 1/4         &           &           \\ \hline
    
    \header{2}{Funkcijų kompozicijos ribos}
    2.1 & $$\lim_{x \to 0} sin({2arctg\frac{1}{x}})$$                                           & 0         &           &           \\ \hline
    2.2 & $$\lim_{x \to 0} arctg(\frac{sin(x)}{cos(x)})$$                                       & 0         &           &           \\ \hline
    2.3 & $$\lim_{x \to 0} arctg(\frac{e^x-1-x}{x^2})$$                                         & 1/2       & 1/2       & \green    \\ \hline
    2.4 & $$\lim_{x \to \frac{\pi}{2}} arctg(\frac{sin(x)}{cos(x)})$$                           & \noLimit  &           &           \\ \hline
    2.5 & $$\lim_{x \to 0} xsin(\frac{1}{x})$$                                                  & 0         &           &           \\ \hline
    2.6 & $$\lim_{x \to 0} \frac{\sin{4x}}{\sin{x}}$$                                           & 0         &           &           \\ \hline
    2.7 & $$\lim_{x \to 0} \frac{cos^2(x)-1)(2*tan(x))}{1-tan^2(x)}$$                           & 0         &           &           \\ \hline

    \end{longtable}
\end{center}