\section{Programos testavimo protokolas}

    Testavimui pasirinkta juodos dežės metodas. 
    Testavimui pasirinkintas įvairus spektras mūsų pačių sugalvotų ribų. 
	Stengtasi padengti visas palaikomas Teiloro skleidinių funkcijas.
	Taip pat, mėginome ištestuoti skirtingas funkcijų kompozicijas.
	
	Žemiau pateiktoje \ref{tab:rez} lentelėje vaizduojami visi testavimo atvejai, sugrupuoti į dvi grupes - 
	elementariųjų funkcijų bei jų kompozicijų ribos. 
	Be to, pateikiamas tikėtini ir gauti ribų atsakimų rezultatai. 
	Jeigu programos gautasis atsakymas sutapo su tikrąja funkcijos riba, 
    rezultatas teigiamas ir pažymėtas žalia spalva, kitu atveju neigiamas - raudona.
    
\begin{center}
    \newcommand*\rot{\rotatebox{90}}
    \newcommand{\x}{\times}
    \newcommand{\green}{\cellcolor{green!50} teigiamas}
    \newcommand{\red}{\cellcolor{red!50} neigiamas}
    \newcommand{\noLimit}{Riba neegzistuoja}
    \newcommand{~}{$\approx$}
    \renewcommand{\inf}{$\infty$}
    \newcommand{\header}[2]{\textbf{#1} & \multicolumn{3}{c}{\textbf{#2}} & \\ \hline}
    \begin{longtable}{|m{0.09\textwidth}|m{0.4\textwidth}|m{0.12\textwidth}|m{0.11\textwidth}|m{0.11\textwidth}|}
    \caption{Testavimo rezultatai} \label{tab:rez} \\
    \hline
        \textbf{Testo numeris} & \textbf{Skaičiuojama riba} &\textbf{Tikėtinas rezultatas} & \textbf{Gautas rezultatas} & \textbf{Rezultatas} \\ \hline
    \endhead
        \multicolumn{5}{|r|}{{Tęsinys kitame puslapyje}} \\ \hline
    \endfoot
    \hline \hline
    \endlastfoot
    \header{0}{Basic}
    0.0  & \[ \lim_{x \to 30} x \]                                                                      & 1             &               &           \\ \hline
    0.1  & \[ \lim_{x \to -1} 1+x \]                                                                    & 1             &               &           \\ \hline
    0.2  & \[ \lim_{x \to \infty} 1-x \]                                                                & -\inf         &               &           \\ \hline
    0.3  & \[ \lim_{x \to 100\pi} cos(x)+sin(x) \]                                                      & 1             &               &           \\ \hline
    0.4  & \[ \lim_{x \to 50} e^{ln x} \]                                                               &50             &               &           \\ \hline
    0.5  & \[ \lim_{x \to \pi} sin(x)cos(x) \]                                                          & 0             &               &           \\ \hline  
    0.6  & \[ \lim_{x \to 0} \frac{ln(x+1)}{x} \]                                                       & 1             &               &           \\ \hline
    0.7  & \[ \lim_{x \to 0} \frac{(1+x)^{42}-1}{x}\]                                                   & 42            &               &           \\ \hline
    0.8  & \[ \lim_{x \to 0} \frac{e^x-1}{x} \]                                                         & 1             &               &           \\ \hline
    0.9  & \[ \lim_{x \to \infty} \frac{ln(x)}{x^7} \]                                                  & 0             &               &           \\ \hline    
    0.10 & \[ \lim_{x \to \infty} \frac{e^x}{x^e} \]                                                    & \inf          &               &           \\ \hline
    0.11 & \[ \lim_{x \to 0} (1+x)^{\frac{1}{x}}\]                                                      & e             &               &           \\ \hline
    0.12 & \[ \lim_{x \to \infty} (1+\frac{1}{x}^x\]                                                   & e             &               &           \\ \hline
    0.13 & \[ \lim_{x \to 0} x\x\frac{1}{x} \]                                                          & 1             &               &           \\ \hline
    0.14 & \[ \lim_{x \to \infty} x\x   \frac{1}{x} \]                                                  & 1             &               &           \\ \hline
    

    \header{1}{Elementarių funkcijų ribos}
    1.1  &\[ \lim_{x \to -1} \frac{1}{1+x} \]                                                           & \noLimit      &               &           \\ \hline
    1.2  & \[ \lim_{x \to 2} \frac{e^{xln2}}{x^2} \]                                                    & 1             &               &           \\ \hline
    1.3  & \[\lim_{x \to 30} \sqrt[5]{x}\]                                                              & $\sqrt[5]{30}$& $\sqrt[5]{30}$& \green    \\ \hline
    1.4  & \[ \lim_{x \to \frac{\pi}{2}} cos(2x) \]                                                     & -1            &               &           \\ \hline
    1.5  & \[ \lim_{x \to 0} e^{\frac{1}{x}} \]                                                         & \noLimit      &               &           \\ \hline
    1.6  & \[ \lim_{x \to 0} e^{\frac{1}{x^2}} \]                                                       & $+\infty$     &               &           \\ \hline  
    1.7  & \[ \lim_{x \to 0} \frac{sin(x)}{x} \]                                                        & 1             &               &           \\ \hline
    1.8  & \[ \lim_{x \to \infty} \frac{sin(x)}{x}\]                                                    & 0             &               &           \\ \hline
    1.9  & \[ \lim_{x \to 0} \frac{sin(x)}{cos(x)} \]                                                   & 0             &               &           \\ \hline
    1.10 & \[ \lim_{x \to 0} \frac{cos(x)}{sin(x)} \]                                                   & \noLimit      &               &           \\ \hline    
    1.11 & \[ \lim_{x \to \infty} \frac{1}{x} \]                                                        & 0             &               &           \\ \hline
    1.12 & \[ \lim_{x \to -\infty} \frac{x^2}{x^4} \]                                                   & 0             &               &           \\ \hline
    1.13 & \[ \lim_{x \to \infty} \frac{x^{\frac{3}{2}}}{x(\sqrt{x+1}+\sqrt{x-1}+2\sqrt{x})} \]         & 1/4           &               &           \\ \hline
    1.14 & \[ \lim_{x \to 0} \frac{e^xsin(x)-x(1+x)}{x^3} \]                                            & 1/3           & 1/3           & \green    \\ \hline
    1.15 & \[ \lim_{x \to 0} \frac{cos(x)-e^{(-\frac{x^2}{2})}}{x^4} \]                                 & -1/12         &               &           \\ \hline
    1.16 & \[ \lim_{x \to 5} (5+x)^5 \]                                                                 & 100000        &               &           \\ \hline
    1.17 & \[ \lim_{x \to 0.7854} \sqrt{2}sin(x) \]                                                     & ~1            &               &           \\ \hline
    1.18 & \[ \lim_{x \to 1} \frac{x^{\frac{3}{2}}}{x(\sqrt{x+1}+\sqrt{x-1}+2\sqrt{x})} \]              & $$\frac{1}{2+\sqrt{2}}$$ ~0.2929 & &      \\ \hline
    1.19 & \[ \lim_{x \to \infty} \frac{x^2}{e^x} \]                                                    & 0             &               &           \\ \hline
    1.20 & \[ \lim_{x \to 0} \frac{\sqrt{x+4}-2}{x} \]                                                  & 1/4           &               &           \\ \hline
    1.21 & \[ \lim_{x \to 0} \frac{\sin{4x}}{\sin{x}}\]                                                 & 4             &               &           \\ \hline
   
    \header{2}{Funkcijų kompozicijos ribos}     
    2.1  & $$\lim_{x \to 0} sin({2arctg\frac{1}{x}})$$                                                  & 0             &               &           \\ \hline
    2.2  & $$\lim_{x \to 0} arctg(\frac{sin(x)}{cos(x)})$$                                              & 0             &               &           \\ \hline
    2.3  & $$\lim_{x \to 0} \frac{arctg(e^x-1-x)}{x^2}$$                                                & 1/2           & 1/2           & \green    \\ \hline
    2.4  & $$\lim_{x \to \frac{\pi}{2}} arctg(\frac{sin(x)}{cos(x)})$$                                  & \noLimit      &               &           \\ \hline
    2.5  & $$\lim_{x \to 0} xsin(\frac{1}{x})$$                                                         & 0             &               &           \\ \hline
    2.6  & $$\lim_{x \to 0} \frac{(cos^2(x)-1)\frac{2sin(x)}{cos(x)}}{1-(\frac{sin(x)}{cos(x)})^2}$$    & 0             &               &           \\ \hline
    2.7  & \[\lim_{x \to 0} \frac{\frac{sin(x)}{x}-1}{sin^2(x-1) + cos^2(x+1) - 1} \]		            & 0				&		        &           \\ \hline
    2.8  & \[\lim_{x \to 0} \frac{sin^2(x-1) + cos^2(x+1) - 1}{\frac{sin(x)}{x}-1} \]		            & \noLimit		&		        &           \\ \hline


    \end{longtable}     
\end{center}