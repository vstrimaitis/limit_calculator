\section{Programos testavimo protokolas}

Testuojama juodosios dežės metodu.


\begin{center}
    \newcommand*\rot{\rotatebox{90}}
    \newcommand{\x}{$\times$}
    \newcommand{\green}{\cellcolor{green!50}}
    \newcommand{\red}{\cellcolor{red!50}}
    \newcommand{\header}[2]{\textbf{#1} & \multicolumn{3}{c}{\textbf{#2}} & \\ \hline}
\
    \begin{longtable}{|m{0.09\textwidth}|m{0.4\textwidth}|m{0.12\textwidth}|m{0.11\textwidth}|m{0.11\textwidth}|}
    \hline
        \textbf{Testo numeris} & \textbf{Skaičiuojama riba} &\textbf{Tiketinas rezultatas} & \textbf{Gautas rezultatas} & \textbf{Rezultatas} \\ \hline
    \endhead
        \multicolumn{5}{|r|}{{Tęsinys kitame puslapyje}} \\ \hline
    \endfoot
    \hline \hline
    \endlastfoot
    \header{1}{Paprastų funkcijų ribos}
    1.1 & lim & 1 & 2  & \green teigiamas \\ \hline
    1.2 & lim & 1 & 2  & \red neigiamas \\ \hline
    

    \end{longtable}
\end{center}