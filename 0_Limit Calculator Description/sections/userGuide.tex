\section{Programos naudotojo vadovas}

	Programos naudojimui sukūrėme internetinį puslapį, kurį vartotojai gali pasiekti 
	naudojantis savo įrenginiu, turinčiu prieigą prie interneto.
	Vartotojo sąsają sudaro keturi pagrindiniai komponentai: funkcijos ir taško įvedimo laukai, ,,go'' mygtukas ir rezultato laukas.
	
	\begin{itemize}
		\item Funkcijos įvedimo laukas -- tai tekstinis laukas, kuriame įvedama vartotojo pasirinkta funckija. Leidžiamos
			naudoti konstantos, funkcijos, ir operatoriai pateikti lentelėje \ref{tab:funcInput}.
		\item Taško įvedimo laukas -- tai tekstinis laukas, kuriame įvedamas vartotojo pasirinktas skaičius į kurį artėja
			kintamojo reikšmė. Galima pateikti baigtinį tašką (leidžiamos naudoti konstantos ir operatoriai pateikti
			lentelėje \ref{tab:pointInput}), arba begalinį -- ,,+inf'' ($+\infty$), ,,-inf'' ($-\infty$), arba ,,inf''
			(tas pats kaip ir ,,+inf'' -- $+\infty$).
		\item Ribos skaičiavimo mygtukas ,,go'' -- tai mygtukas, kurį paspaudus siunčiamas kreipimasis į serverį ir laukiama skaičiavimo rezultato.
		\item Rezultato laukas -- jame vaizduojamas gautas skaičiavimo rezultatas. 
		Jį sudaro gauta ribos reikšmė arba klaidos pranešimas, nusakantis kodėl nepavyko jos apskaičiuoti.
	\end{itemize}
	
	Ribų skaičiavimo sistema pateikia vieną iš keturių atsakymų, kurį parodo internetinis puslapis:

	\begin{itemize}
		\item Jei riba egzistuoja ir programa ją rado, grąžinama rasta riba.
		\item Jei riba neegzistuoja ir programa, pranešama, kad riba neegzistuoja.
		\item Jei įvesta funkcija arba taškas yra sintaksiškai neteisingi, apie tai pranešama ir parodoma klaidos vieta.
		\item Jei programa negalėjo nustatyti ribos arba ar riba išvis egzistuoja, pranešama, kad nepavyko išanalizuoti ribos.
	\end{itemize}
	
	\begin{table} [H]
		\renewcommand{\tabularxcolumn}[1]{m{#1}}
		\centering
		\caption{Funkcijos lauko įvestis}
		\label{tab:funcInput}
		\begin{tabular} {| l | l |}
			\hline
			Matematinis reiškinys					& Atitikmuo sistemoje 			\\
			\hline		
			Suma 									& +								\\
			\hline					
			Atimtis 								& -								\\
			\hline
			Daugyba									& *								\\
			\hline
			Dalyba 									& 	/							\\
			\hline
			Kėlimas laipsniu 						& \string^							\\
			\hline
			Skliaustai 								& ( ir )						\\
			\hline	
			Sinusas 								& sin  							\\
			\hline	
			Kosinusas								& cos 							\\
			\hline
			Arktangentas							& arctg arba atan				\\
			\hline
			Natūrinis logoritmas					& ln 							\\
			\hline
			Kvadratinė šaknis						& sqrt 							\\
			\hline
			Konstantos e kėlimas laipsniu			& exp 							\\
			\hline
			$\pi$									& pi							\\
			\hline	
			e 										& e								\\
			\hline
		\end{tabular}
	\end{table}
	
	
	\begin{table} [H]
		\renewcommand{\tabularxcolumn}[1]{m{#1}}
		\centering
		\caption{Taško lauko įvestis}
		\label{tab:pointInput}
		\begin{tabular} {| l | l |}
			\hline
			Matematinis reiškinys					& Atitikmuo sistemoje 			\\
			\hline		
			$\pi$									& pi							\\
			\hline	
			e 										& e								\\
			\hline	
			Suma 									& +								\\
			\hline					
			Atimtis 								& -								\\
			\hline
			Daugyba									& *								\\
			\hline
			Dalyba 									& 	/							\\
			\hline
			Kėlimas laipsniu 						& \string^	 							\\
			\hline
			Skliaustai 								& ( ir )						\\
			\hline
		\end{tabular}
	\end{table}
	
	
	