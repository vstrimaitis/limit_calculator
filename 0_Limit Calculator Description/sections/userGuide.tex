\section{Programos naudotojo vadovas}

	Programos naudojimui sukūrėme internetinį puslapį, kurį vartotojai gali pasiekti 
	naudojantis savo įrenginiu, turinčiu prieigą prie interneto.
	Vartotojo sąsają sudaro keturi pagrindiniai komponentai: funkcijos ir taško įvedimo laukai, ,,='' mygtukas ir rezultato laukas.
	
	\begin{itemize}
		\item funkcijos įvedimo laukas - tai tekstinis laukas, kuriame įvedama vartotojo pasirinkta funckija.
		\item taško įvedimo laukas - tai tekstinis laukas, kuriame įvedamas vartotojo pasirinktas skaičius į kurį artėja kintamojo reikšmė.
		\item ribos skaičiavimo mygtukas ,,='' - tai mygtukas, kurį paspaudus siunčiamas kreipimasis į serverį ir laukiamas skaičiavimo rezultatas.
		\item rezultato laukas - tekstinis laukas, kuriame vaizduojamas gautasis skaičiavimo rezultatas. 
		Jį sudaro gauta ribos reikšmė arba klaidos pranešimas, nusakantis kodėl nepavyko jos apskaičiuoti.
	\end{itemize}
	
	
	
	\begin{table} [H]
		\renewcommand{\tabularxcolumn}[1]{m{#1}}
		\centering
		\caption{Funkcijos lauko įvestis}
		\label{tab:funcInput}
		\begin{tabular} {| l | l |}
			\hline
			Matematinis reiškinys					& Atitikmuo sistemoje 			\\
			\hline		
			Suma 									& +								\\
			\hline					
			Atimtis 								& -								\\
			\hline
			Daugyba									& *								\\
			\hline
			Dalyba 									& 	/							\\
			\hline
			Kėlimas laipsniu 						& \string^							\\
			\hline
			Skliaustai 								& ( ir )						\\
			\hline	
			Sinusas 								& sin  							\\
			\hline	
			Kosinusas								& cos 							\\
			\hline
			Arktangentas							& arctg arba atan				\\
			\hline
			Natūrinis logoritmas					& ln 							\\
			\hline
			Kvadratinė šaknis						& sqrt 							\\
			\hline
			Konstantos e kėlimas laipsniu			& exp 							\\
			\hline
			$\pi$									& pi							\\
			\hline	
			e 										& e								\\
			\hline
		\end{tabular}
	\end{table}
	
	
	\begin{table} [H]
		\renewcommand{\tabularxcolumn}[1]{m{#1}}
		\centering
		\caption{Taško lauko įvestis}
		\label{tab:pointInput}
		\begin{tabular} {| l | l |}
			\hline
			Matematinis reiškinys					& Atitikmuo sistemoje 			\\
			\hline		
			$\pi$									& pi							\\
			\hline	
			e 										& e								\\
			\hline	
			Suma 									& +								\\
			\hline					
			Atimtis 								& -								\\
			\hline
			Daugyba									& *								\\
			\hline
			Dalyba 									& 	/							\\
			\hline
			Kėlimas laipsniu 						& \string^	 							\\
			\hline
			Skliaustai 								& ( ir )						\\
			\hline
		\end{tabular}
	\end{table}
	
	
	