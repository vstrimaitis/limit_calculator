\section{Sistemos architektūra}

Programos kūrimui pasirinkta Haskell programavimo kalba bei Stack projektų valdymo įrankis.
Kūrimui buvo pasirinkta Haskell programavimo kalba dėl šių priežasčių:
\begin{enumerate}
    \item Tingus skaičiavimas (angl. \textit{Lazy evaluation}). Haskell programavimo kalboje galima aprašyti begalines
        duomenų struktūras, o veikimo metu jų sukonstruojama tik tokia dalis, kokia reikalinga, norint rasti atsakymą.
        Pavyzdžiui, galime aprašyti, kad reiškinys turi begalinį Teiloro skleidinį, tačiau jį spausdinant reikalingas
        tik pirmasis narys, todėl tik vienas narys ir bus suskaičiuotas.
    \item Stipri tipų sistema. Haskell tipų sistema leidžia rašyti lengvai komponuojamą ir pernaudojamą kodą. Pavyzdžiui,
        skaičiavimą galima pritaikyti taip, kad būtų naudojami tiek apytiksliai, tiek visiškai tikslūs skaičiai. Taip pat
        nesunku valdyti skaičiavimo metu vykstančius efektus, pavyzdžiui, kai randame, kad kažkokia funkcijos dalis yra
        neapibrėžta, arba neturime informacijos žingsnio atlikimui ir būtinai turime viską nutraukti.
    \item Funkcinė programavimo kalba iš prigimties tinkama matematinių skaičiavimų modeliavimui.
\end{enumerate}

Siekiant palengvinti funkcionalumo perpanaudojamumą, sistemą suskirstyta į keturis modulius, kurių kiekvienas atlieka
tarpusavyje nesusijusias funkcijas. Sistema moduliai yra šie:
\begin{enumerate}
    \item Ribų skaičiavimo modulis. Tai -- pagrindinė sistemos dalis, kuri atsakinga už pačių ribų skaičiavimą.
    \item Sintaksinės analizės modulis (angl. \textit{parser}). Šis modulis atlieka vertimą iš vartotojo įvesto teksto į
        ribų skaičiavimo moduliui suprantamą duomenų formatą.
    \item Internetinio serverio modulis. Šis modulis leidžia kviesti biblioteką naudojant REST API bei pateikia vartotojo
        sąsają, kad vartotojams ją būtų galima pasiekti naršyklėje.
    \item Vartotojo sąsajos modulis. Tai -- internetinis puslapis, sąveikaujantis su anksčiau minėtu internetiniu serveriu,
        kad vartotojas visą funkcionalumą galėtų pasiekti naršyklėje.
\end{enumerate}
