\section{Sistemos architektūros projektas}

Programos kūrimui pasirinkta Haskell funkcinė programavimo kalba.
Programos modulių rašymui sukurtas Stack projektas. 
Būtent ši funkcinė programavimo kalba pasirinkta vietoj procedūrinės ar objektinės dėl kelių priežasčių, kaip: 
savo tingaus skaičiavimo implementacijos (angl. \textit{Lazy computation}) ir stipri tipų sistema. 
 
Mūsų kuriamą sistemą sudaro: ribų skaičiavimo biblioteka, įvesties funkcijos sintaksinės analizės ir vartotojo sąsajos moduliai. 
Vartotojo sąsajos modulis atsakingas, jog vartotojas galėtų į programą konsolės sąsajoje 
įvesti pasirinktą funkciją, kurios ribą nori apskaičiuoti. 
Taip pat, vartotojui pateikiamas ir gautas apskaičiuotas funkcijos ribos atsakymas, jeigu jį programai gauti pavyko. 
Kitu atveju, pateikiama, jog riba yra ne skaičius (teigiama ar neigiama begalybė), ji neegzistuoja arba apskaičiuoti nepavyko.
Įvesties analizės modulis atlieka įvesties formato funkcijos konvertavimo į 
patogią ribų skaičiavimui programoje tinkamą formą (angl. \textit{parse}). 
Ribų skaičiavimo biblioteka atsakinga, už įvestos funkcijos skleidimą pasinaudojant Teiloro skleidiniais ir
mūsų apibrėžtomis - Laipsnių eilutėmis ir euristikomis.

%\[ \lim_{x \to 0} \frac{\frac{sin(x)}{x}-1}{sin^2(x-1) + cos^2(x+1) - 1} \]		& 0				&		& // /hline

%\[ \lim_{x \to 0} \frac{sin^2(x-1) + cos^2(x+1) - 1}{\frac{sin(x)}{x}-1} \]		& \noLimit		&		& // /hline