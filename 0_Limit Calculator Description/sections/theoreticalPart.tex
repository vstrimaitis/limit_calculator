\section{Teorinės dalies aprašymas}

%\subsection*{Laipsnių eilutės}

%\[\sum_{n=0}^{\infty} a_n(x_n)^n = a_0 + a_1 x + a_2 x^2 + ...\]

%\[\sum_{m=1}^{k} b_m(x_m)^{-m} = b_1 x^{-1} + b_2 x^{-2} + ... + b_k x^{-k}\]

%\[c_n x^n = a_n(x_n)^n + a_{n-1}x^{n-1} \times b_{1}{x} + a_{n+1}(x_{n+1}) \times b_{-1}x^{-1} + ...\]


Programos kūrimo metodika pasirinkta laipsnių eilutės. 
Tai yra Teiloro eilučių praplėtimas, kuriame leidžiame turėti baigtiniį kiekį neigiamų laipsnių. 
Jeigu skleidžiant įvestį gaunamas begalinis kiekis neigiamų laipsnių, ši metodika nebėra tinkama taikyti norint rasti išraiškos ribą.
Tokiu atveju, naudojame eurisitinę analizę. 

\subsection{Teiloro eilutės}

	Teiloro skleidiniai yra būdas aproksimuoti funkcijų reikšmes artėjant į konkretų tašką.
	Pagrindinė apytikslės reikšmės skaičiavimo sąlyga - funkcija privalo būti diferencijuojama norimame intervale.
	Funkcija keičiama polinomu pasinaudojant teiloro skleidiniu ir pasirinkto taško aplinkoje ji elgiasi pakankamai tiksliai kaip tikroji funkcija.
	Kadangi skleidinio forma yra begalinė, tikslumą galima laisvai pasirinkti, imant norimą kiekį polinomo narių. 
	
	
	\subsection*{Teiloro begalinis skleidinys}

\[\lim_{x \to a} f(x) = f(a) + \frac{f'(a)}{1!}(x-a)+\frac{f''(a)}{2!}(x-a)^2 + \frac{f'''(a)}{3!}(x-a)^3 + ...\]

\[\lim_{x \to a} f(x) = \sum_{n=0}^{\infty} \frac{f^{(n)}(a)}{n!}(x-a)^n\]

\subsection*{Teiloro baigtinis skleidinys}

\[\lim_{x \to a} f(x) = f(a) + \frac{f'(a)}{1!}(x-a)+\frac{f''(a)}{2!}(x-a)^2 + ... + \frac{f^{(n)}(a)}{n!}(x-a)^n + r_n(x) \]

\[\lim_{x \to a} f(x) = \sum_{k=0}^{n} \frac{f^{(k)}(a)}{k!}(x-a)^k + r_n(x)\]

Tegu funkcija $f$ yra $n+1$ kartą diferencijuojama.
Tada:

\[r_n(x) = \frac{f^{(n+1)}(c)}{(n+1)!}(x-a)^{n+1} \textrm{ , kur c } \in (a,x) \textrm{ arba c } \in (x,a).\]

Jei $f^{(n)}$ tolydi, tai:

\[ r_n(x) = o((x-a)^n) \textrm{ , kai } x \to a.\]

\subsection*{Elementarių funkcijų ribų pavyzdžiai}

Pateikiami funcijų skleidiniai taške $a$ ir $a=0$. 

$\bm{e^x}$ \textbf{riba}

\[\lim_{x \to a} e^x= e^a + \frac{e^a}{1!} (x-a) + \frac{e^a}{2!}(x-a)^2 + \frac{e^a}{3!}(x-a)^3 + ... \]

Kai $a = 0$, tai: 
\[\lim_{x \to 0} e^x= e^0 +  \frac{e^0}{1!} (x-0) + \frac{e^0}{2!} (x-0)^2 + \frac{e^0}{3!} (x-0)^3 + ... = 1 + x + \frac{x^2}{2} + \frac{x^3}{6} + ... \]

$\bm{sin(x)}$ \textbf{riba}

\[ \lim_{x \to a} sin(x) = sin(a) + \frac{cos(a)}{1!}(x-a)+\frac{-sin(a)}{2!}(x-a)^2+\frac{-cos(a)}{3!}(x-a)^3+\frac{sin(a)}{4!}(x-a)^4+...\]

Kai $a = 0$, tai:
\[ \lim_{x \to 0} sin(x) = sin(0) + \frac{cos(0)}{1!}(x-0)+\frac{-sin(0)}{2!}(x-0)^2+\frac{-cos(0)}{3!}(x-0)^3+\frac{sin(0)}{4!}(x-0)^4 + ... = \] 

\[ = 0 + x + 0 + \frac{-1}{3!}x^3 + 0 + \frac{1}{5!}x^5 + ... = x - \frac{x^3}{6}+\frac{x^5}{120} - ... \]

$\bm{cos(x)}$ \textbf{riba}

\[ \lim_{x \to a} cos(x) = cos(a) + \frac{-sin(a)}{1!}(x-a)+\frac{-cos(a)}{2!}(x-a)^2+\frac{sin(a)}{3!}(x-a)^3+\frac{cos(a)}{4!}(x-a)^4+...\]

Kai $a = 0$, tai:
\[ \lim_{x \to 0} cos(x) = cos(0) + \frac{-sin(0)}{1!}(x-0)+\frac{-cos(0)}{2!}(x-0)^2+\frac{sin(0)}{3!}(x-0)^3+\frac{cos(0)}{4!}(x-0)^4 + ... = \] 

\[ = 1 + 0 + \frac{-1}{2!}x^2 + 0 + \frac{1}{4!}x^4 + ... = 1 - \frac{x^2}{2}+\frac{x^4}{24} - ... \]

$\bm{(1+x)^k}$ \textbf{riba}

\[\lim_{x \to a} (1+x)^k =\]
\[=(1+x)^k + \frac{k(1+a)^{k-1}}{1!} (x-a) + \frac{k(k-1)(1+a)^{k-2}}{2!}(x-a)^2 + \frac{{k(k-1)(k-2)(1+a)^{k-3}}}{3!}(x-a)^3 + ... \]

Kai $a = 0$, tai: 

\[\lim_{x \to 0} (1+x)^k =\]
\[=(1)^k + \frac{k(1)^{k-1}}{1!} (x-0) + \frac{k(k-1)(1)^{k-2}}{2!}(x-0)^2 + \frac{{k(k-1)(k-2)(1)^{k-3}}}{3!}(x-0)^3 + ...  = \]
\[=1 + kx + \frac{k(k-1)}{2}x^2 + \frac{{k(k-1)(k-2)}}{6}x^3 + ...  = \]

$\bm{ln(1+x)}$ \textbf{riba}

\[ \lim_{x \to a} ln(1+x) = ln(1+a) + \frac{\frac{1}{1+a}}{1!}(x-a) + \frac{-\frac{1}{(1+a)^2}}{2!}(x-a)^2 + \frac{\frac{2}{(1+a)^3}}{3!}(x-a)^3 + ... \]

Kai $a = 0$, tai:

\[ \lim_{x \to 0} ln(1+x) = ln(1+0) + \frac{\frac{1}{1+0}}{1!}(x-0) + \frac{-\frac{1}{(1+0)^2}}{2!}(x-0)^2 + \frac{\frac{2}{(1+0)^3}}{3!}(x-0)^3 + ... = \]

\[ = 0 + x + \frac{-1}{2}x^2 + \frac{1}{3}x^3 + ... = x - \frac{x^2}{2} + \frac{x^3}{3} - ...\]

$\bm{\arctan(x)}$ \textbf{riba}

\[\lim_{x \to a} \arctan(x) = \arctan(a) + \frac{\frac{1}{a^2+1}}{1!}(x-a) + \frac{\frac{-2a}{(a^2+1)^2}}{2!}(x-a)^2 + \frac{\frac{6a^2-2}{(a^2+1)^3}}{3!}(x-a)^3 \]

Kai $a = 0$, tai: 

\[\lim_{x \to 0} \arctan(x) = \arctan(0) + \frac{\frac{1}{0^2+1}}{1!}(x-0) + \frac{\frac{-2*0}{(0^2+1)^2}}{2!}(x-0)^2 + \frac{\frac{6*0^2-2}{(0^2+1)^3}}{3!}(x-0)^3 =\]
\[= 0 + x + 0 + \frac{1}{3} x^3 +... = x - \frac{x^3}{3} + \frac{x^5}{5} - ... \]

	
\subsection{Laipsnių eilutės}
	
	Laipsnių eilutės užrašome taip:
	
	\[\sum_{n=0}^{\infty} a_nx^n + \sum_{m=-1}^{-k} a_mx^m = \sum_{i=-k}^{\infty} a_ix^i  = a_{-k} x^{-k} + ... + a_{-1} x^{-1} + a_0 + a_1 x + ... \]
	
	Tokiame užraše matome, jog teigiamų laipsnių galime turėti be galo daug, o neigiamų laipsnių kiekis būtinai turi būti baigtinis.
	
	Šioms eilutėms galima taikyti aritmetines operacijas:
	
	\begin{itemize}
		\item Sudėties operacija atliekame sudėdami atitinkamus eilučių koeficientus ir gauname tokios pačios formos eilutę.
		\item Atimties operacija veikia atitinkamai sudėčiai, tai yra, koeficientai atimami.
		\item Daugybos opercija atliekama taip - kiekvieną pirmos eilutės narį dauginame su antros eilutės nariu.
			  Gautą išraišką sutraukiame ir kiekvienam konkrečiam $n$ yra baigtinis skaičius narių turinčių $x^n$.
			  Taip yra, nes $x^n$ išraišką gauname sudauginę $x^i$ ir $x^j$, kur $i <= \frac{n}{2}$ ir $j >= \frac{n}{2}$ ir atvirkščiai.
			  Kiekvieną koeficientas yra konkretus skaičius, nes yra sutrauktas iš baigtinio kiekio narių.
		\item Dalybos operacija atliekama pasiremiant polinomų dalybos stulpeliu algoritmą. 
			  Šis algoritmas veikia taip pat kai turime ir baigtinius, ir begalinius dalinio ir daliklio argumentus.
	\end{itemize}
	
	Turint aukščiau įvardintas operacijas galima bet kokią iš jų sudarytą išraišką pasiversti laipsnių eilute.
	Dabar galima apibrėžti elgseną su funkcijomis.
	
	Sakykime turime išraišką $f(S)$ kur S yra laipsnių eilutė. 
	Imame šios funkcijos Teiloro skleidinį kažkokiame taške $a$: 
	
	\[f(x) = f(a) + \frac{f'(a)}{1!}(x)+\frac{f''(a)}{2!}(x)^2 + +\frac{f'''(a)}{3!}(x)^3 + ...\]
	
	Jeigu $S$ turi bent vieną neigiamą $x$ laipsnį, tai keldami S visais natūraliaisiais laipsniais, galutinėje išraiškoje gausime begalo daug skirtingų neigiamų x laipsnių.
	Tokia išraiška negali būti aprašyta mūsų apibrėžtomis laipsnių eilutėmis, nes neigiamų laipsnių privalo būti baigtinis kiekis. 
	Todėl šioms situacijoms spręsti mes pasitelksime euristika.
	Jeigu $S$ neturi nė vieno neigiamo $x$ laipsnio, galime imti $a = a_0$ ($S$ koeficientas prie $x^0$) .
	Pasižymime:
	
	$P = \frac{S - a_0}{x}$
	
	Tada turime: 
	
	$S - a_0 = P \times x$
	
	Perrašome skleidinį:
	
	\[f(a) + \frac{f'(a)}{1!}(Px)+\frac{f''(a)}{2!}(Px)^2 + +\frac{f'''(a)}{3!}(Px)^3 + ...\]
	
	Galutinei laipsnių eilutei, į kurią susiprastina šis reiškinys, norime surasti koeficientą prie $x^n$. 
	Šiame reiškinyje matome, kad šis koeficientas priklauso tik nuo pirmųjų $n+1$ sumos narių. 
	Todėl norėdami rasti šį koeficientą, mes paimame tuos $n+1$ narių, juos suprastiname (naudodamiesi aukščiau parodytomis aritmetinėmis operacijomis) 
	ir iš gautos eilutės paimame koeficientą prie $x^n$. 
	Iš to seka, jog šis reiškinys susiprastina į laipsnių eilutę - galime rasti bet kurį tos eilutės koeficientą.
	
	Savo programoje galime pridėti naujų funkcijų, tereikia pateikti, kaip skaičiuojami tų funkcijų Teiloro skleidinio koeficientai. 
	Tai yra, turime pateikti funkciją: 
	
	\[ g(a,n) = \frac{f^{(n)}(a)}{n!}\]
	
\subsection{Euristikos}
	
	%Toliau euristikos bus
	
	